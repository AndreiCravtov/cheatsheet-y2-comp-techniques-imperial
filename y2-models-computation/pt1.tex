
\subsection*{General - Uncategorized}

\stu{Common features} of definitions of \kwub{algorithm}:
\begin{enumerate}
  \vItem \stb{finite} description of the procedure in terms of elementary operations
  \vItem \stb{deterministic} => next step is uniquely determined if there is one
  \vItem procedure may not terminate on some input data, but we can recognise when it does terminate and \stu{what} the \stu{result} will be
  \vItem We can think of \stu{algorithms} as \kwb{computable functions} \textit{(e.g. RM-computable, TM-computable, etc.)}
\end{enumerate}

\hSep

\kwub{Partial function} \iMbox{f: X \rightharpoonup Y} is such that:
\begin{enumerate}
  \vItem \iMbox{X \rightharpoonup Y} is the \stu{set of partial functions} from \iMbox{X} to \iMbox{Y}
  \vItem \iMbox{f(x) = y} means \iMbox{(x, y) \in f}
  \vItem \iMbox{f(x) = y \ \wedge \ f(x) = y^{\prime} \ \ \implies \ \ y=y^{\prime} }
  \vItem \iMbox{f(x) \downarrow} means \iMbox{\exists y \in Y(f(x)=y)} \ ; \
  \iMbox{f(x) \uparrow} means \iMbox{\neg{f(x) \downarrow}}
  \vItem \iMbox{f} is \kwub{total} if \iMbox{\forall x \in X} we have \iMbox{f(x) \downarrow}
  \vItem \iMbox{X \to Y} is \stu{set of total functions} from \iMbox{X} to \iMbox{Y}
\end{enumerate}


\hSep

\kwub{General halting problem} is the decision problem with:
\begin{enumerate}
  \vItem the set \iMbox{S} of all pairs \iMbox{(A, D)}, where \iMbox{A} is an \stb{algorithm} and
  \iMbox{D} is some \stb{input datum} on which \iMbox{A} is designed to operate
  \vItem \iMbox{A(D) \downarrow} holds for \iMbox{(A, D) \in S} if \iMbox{A} applied to \iMbox{D} \stu{eventually halts}
  \vItem The \kwu{halting problem} is \stb{unsolvable (undecidable)}, i.e.
  no algorithm \iMbox{H: S \to \mathbb{B}} exists s.t. \iMbox{H(A,D) = \begin{cases}
      \mathbf{tt} & A(D) \downarrow \\
      \mathbf{ff} & A(D) \uparrow
    \end{cases}}
\end{enumerate}

\hSep

\subsection*{Register Machine Definitions}

A \kwub{register machine (RM)} is specified by:
\begin{itemize}
  \vItem finite \stb{registers} \iMbox{R_{0}, \dots, R_{n}}; each can store one nat. \iMbox{\mathbb{N}}
  \vItem a \stb{program} => finite list of \stu{instructions}

  \begin{enumerate}
    \vItem each \stb{instruction} of the form \iMbox{L_i : B_i}
    \vItem \iMbox{L_i} is \stb{label} of \iMbox{i+1}-th \stu{instruction},
    for \iMbox{i = 0,1,\dots}
    \vItem \iMbox{B_i} is \stb{instruction body}, and is one of:

    \begin{enumerate}
      \vItem \iMbox{R^{\textcolor{cornellred}{+}} \rightarrow L^{\prime}} =>
      add \iMbox{1} to \stu{reg.} \iMbox{R} \& jump to \stu{inst.} \iMbox{L^{\prime}}
      \vItem \iMbox{R^{\textcolor{cornellred}{-}} \rightarrow L^{\prime}, L^{\prime \prime}} =>
      \stub[orange]{if} \iMbox{R > 0} sub. \iMbox{1} from \iMbox{R} \& jump to \iMbox{L^{\prime}};
      \stub[orange]{else} jump to \iMbox{L^{\prime \prime}}
      \vItem \iMbox{HALT} => stop executing instructions
    \end{enumerate}
  \end{enumerate}

  \vItem \stu{initial/final} reg. contents related by \stu{partial function}
\end{itemize}

\hSep

\kwub{RM configuration} => \iMbox{c=\left(\ell, r_{0}, \dots, r_{n}\right)}
\begin{enumerate}
  \vItem \iMbox{\ell = \text{current label}}; \iMbox{r_i = \text{current contents of} \ R_i}
  \vItem \iMbox{R_{i}=x[\text{in config.} \ c]} means
  \iMbox{c=\left(\ell, r_{0}, \dots, r_{n}\right)} with \iMbox{r_{i}=x}
  \vItem \kwb{initial configuration} => \iMbox{c_{0}=\left(0, r_{0}, \dots, r_{n}\right)}
\end{enumerate}

\hSep

\kwub{RM computation} => finite \textit{(or infinite)} \stu{sequence} of \stu{configurations} \iMbox{c_{0}, c_{1}, c_{2}, \dots}
\begin{enumerate}
  \vItem \iMbox{c_{0}=\left(0, r_{0}, \dots, r_{n}\right)} is \stu{initial configuration}
  \vItem each \iMbox{c_{i+1}} determined from \iMbox{c_i =\left(\ell, r_{0}, \dots, r_{n}\right)}
  by performing \stu{inst.} labelled \iMbox{L_{\ell}}
  \vItem \kwb{halting computation} => \stu{finite sequence} \iMbox{c_{0}, c_{1} \dots, c_{m}}
  \begin{enumerate}
    \vItem \kwb{halting configuration} \iMbox{c_{m}=(\ell, r, \dots)}

    \vItem \stu{inst.} labelled \iMbox{L_{\ell}} is
    \stub[orange]{either} \iMbox{HALT} i.e. \kwb{proper halt};
    \stub[orange]{or} performs \stu{jump to nonexistent label}  i.e. \kwb{erroneous halt}

    \vItem e.g. \iMbox{\begin{aligned}
        L_{0} & : R_{1}^{+} \rightarrow L_{2} \\
        L_{1} & : HALT
      \end{aligned}} \stu{halts erroneously}
  \end{enumerate}

  \vItem \kwb{Non-halting Computation} => \stu{infinite sequence} \iMbox{c_{0}, c_{1}, \dots},
  e.g. \iMbox{\begin{aligned}
      L_{0} & : R_{1}^{+} \rightarrow L_{0} \\
      L_{1} & : HALT
    \end{aligned}}
\end{enumerate}

\hSep

\kwub{RM graphical representation}
\begin{enumerate}
  \vItem \stu{one node} per \stu{inst.} \iMbox{L_\ell : B_\ell}
  \vItem each node labelled by \iMbox{[L_\ell]}, where \iMbox{[L_\ell]} is \stu{register} of
  \stu{inst. body} \iMbox{B_\ell}
  \vItem \kwb{arcs} are \stu{jumps}; \stb{initial instruction} is \iMbox{START}
\end{enumerate}
\mkImg[70pt]{image}

\hSep

\iMbox{f: \mathbb{N}^n \rightharpoonup \mathbb{N}} is \kwub{(RM) computable function} if:
\begin{enumerate}
  \vItem There is RM \iMbox{M} with \iMbox{\geq n + 1} registers, such that
  \vItem for all \iMbox{(x_1,\dots, x_n)\in \mathbb{N}^n, y \in \mathbb{N}} we have
  \stu{initial configuration} \iMbox{c_0 = (0, x_1, \dots, x_n, 0, 0, \dots)}, and
  \vItem \iMbox{M} halts with \iMbox{R_0 = y} \stub[orange]{iff} \iMbox{f(x_1,\dots, x_n) = y}
  \vItem \stu{Basic computable functions} => \stb{addition} ; \stb{multiplication} ;
  \stb{projection} \iMbox{(x,y) \mapsto x} ; \stb{constant} \iMbox{x \mapsto k} ;
  \stb{truncated subtraction} \iMbox{(x,y) \mapsto \max(0, x-y)} ;
  \stb{integer division} \iMbox{(x,y) \mapsto \begin{cases}
      \lfloor x / y \rfloor & y > 0 \\
      0                     & y = 0
    \end{cases}} ;
  \stb{integer remainder} \iMbox{(x,y) \mapsto \begin{cases}
      x - y \cdot \lfloor x / y \rfloor & y > 0 \\
      x                                 & y = 0
    \end{cases}} ;
  \stb{exp. and log. base-2} \iMbox{x \mapsto 2^x, \ x \mapsto \begin{cases}
      \lfloor \log_2(x) \rfloor & x > 0 \\
      0                         & x = 0
    \end{cases}}
\end{enumerate}

\hSep

\subsection*{Numerical Coding}

\stub{Numerical Coding of Pairs} => for \iMbox{\alpha,\beta \in \mathbb{N}} define:
\begin{enumerate}
  \vItem \iMbox{\lAngle \alpha,\beta \rAngle \ \triangleq \ 2^\alpha (2\beta + 1)}
  is a \stu{bijection} between \iMbox{\mathbb{N}^2} and \iMbox{\mathbb{N}^+}
  \begin{enumerate}
    \vItem where \iMbox{\mathbb{N}^+ = \{ n \in \mathbb{N} | n > 0 \}}
    \vItem \stu{Binary form} => \iMbox{\mathrm{0b} \lAngle \alpha,\beta \rAngle =
      \mathrm{0b} \left[ \beta \middle| 1 \middle| \underbrace{0 \cdots 0}_{\alpha} \right]}
  \end{enumerate}

  \vItem \iMbox{\langle \alpha,\beta \rangle \ \triangleq \ \lAngle \alpha,\beta \rAngle - 1 = 2^\alpha (2\beta + 1) - 1}
  \begin{enumerate}
    \vItem \iMbox{\langle —, — \rangle: \mathbb{N}^2 \to \mathbb{N}} is a \stu{bijection} between \iMbox{\mathbb{N}^2} and \iMbox{\mathbb{N}}
    \vItem \stu{Binary form} => \iMbox{\mathrm{0b} \langle \alpha,\beta \rangle =
      \mathrm{0b} \left[ \beta \middle| 0 \middle| \underbrace{1 \cdots 1}_{\alpha} \right]}
  \end{enumerate}

  \vItem \textbf{EXAMPLE:} \iMbox{27 = \mathrm{0b}11011 = \lAngle 0, 13 \rAngle = \langle 2, 3 \rangle}
\end{enumerate}

\hSep

\stub{Numerical Coding of Lists}:
\begin{enumerate}
  \vItem \iMbox{List \ \mathbb{N}} is \ul{set of all finite lists} of \stu{natural numbers}:
  \begin{enumerate}
    \vItem \kwb{empty list} => \iMbox{[] \in {List \ \mathbb{N}}}
    \vItem \kwb{list cons} => \iMbox{{x {::} \ell} \in {List \ \mathbb{N}} \ \ 
    \impliedby \ \ x \in \mathbb{N} \ \wedge \ \ell \in {List \ \mathbb{N}}}
    \vItem \kwb{notation} => \iMbox{[x_{1}, x_{2}, \dots, x_{n}] \triangleq
     {x_{1} {::} (x_{2} {::} (\cdots {x_{n} {::} []} \cdots))}}
  \end{enumerate}
  \vItem Define \stu{bijection} \iMbox{{\tlcorner — \trcorner}: {List \ \mathbb{N}} \to \mathbb{N}} inductively:
  \begin{enumerate}
    \vItem \iMbox{{\tlcorner [] \trcorner} \ \triangleq \ 0} \ ; \ 
    \iMbox{{\tlcorner x {::} \ell \trcorner} \ \triangleq \ \lAngle x, {\tlcorner \ell \trcorner} \rAngle
    = 2^x (2{\tlcorner \ell \trcorner} + 1)}
    \vItem \stu{Binary form} => \iMbox{\mathrm{0b} {\tlcorner [x_1, \dots, x_{n-1}, x_n] \trcorner} =
      \mathrm{0b} \left[ 
        1 \underbrace{0 \cdots 0}_{x_n} \middle|
        1 \underbrace{0 \cdots 0}_{x_{n-1}} \middle| \cdots \middle|
        1 \underbrace{0 \cdots 0}_{x_1}
      \right]}
  \end{enumerate} 
\end{enumerate}

\hSep

Let \iMbox{P = [L_0{:}B_0, L_1{:}B_1, \dots, L_n{:}B_n]} be an \stb{RM program}:
\begin{enumerate}
  \vItem \stub{Numerical Coding of Bodies} \iMbox{{\tlcorner B \trcorner}} is defined by \iMbox{\begin{cases}
    {\tlcorner R_i^{\textcolor{cornellred}{+}} \rightarrow L_j \trcorner}
    &\triangleq \ \lAngle 2i, j \rAngle \\
    {\tlcorner R_i^{\textcolor{cornellred}{-}} \rightarrow L_j, L_k \trcorner}
    &\triangleq \ \lAngle 2i+1, \langle j,k \rangle \rAngle \\
    {\tlcorner HALT \trcorner} &\triangleq \ 0
  \end{cases}} and \iMbox{{\tlcorner B \trcorner}} defines a \stu{bijection}
  \vItem \stub{Numerical Coding of RM Programs} \iMbox{{\tlcorner P \trcorner} \ \triangleq \ 
  {\tlcorner [{\tlcorner B_0 \trcorner}, {\tlcorner B_1 \trcorner}, \dots, {\tlcorner B_n \trcorner}] \trcorner}}
  defines a \stu{bijection}

  \vItem Any \iMbox{x \in \mathbb{N}} decodes to \stu{unique instruction} \iMbox{body(x)}
  \begin{enumerate}
    \vItem \iMbox{x = 0 \ \implies \ body(x) = HALT}
    \vItem \iMbox{x > 1 \ \implies \ x = \lAngle y, z \rAngle} and:
    \begin{enumerate}
      \vItem \iMbox{y = 2i \ \implies\ body(x) = R_i^{\textcolor{cornellred}{+}} \rightarrow L_z}
      \vItem \iMbox{y = 2i + 1 \ \implies \ z = \langle j,k \rangle} and
      \iMbox{body(x) = R_i^{\textcolor{cornellred}{-}} \rightarrow L_j,L_k}
    \end{enumerate}
  \end{enumerate}

  \vItem Any \iMbox{e \in \mathbb{N}} decodes to \stu{unique program} \iMbox{prog(e)}
  called \stub{program with index} \iMbox{e}
\end{enumerate}

\hSep

\subsection*{Register Machine Gadgets}

\kwub{RM gadgets} => \stu{partial RM graph} with \stb{$1$ entry wire} \& \stb{$\geq 1$ more exit wires}
\begin{enumerate}
  \vItem \stu{Input/output registers} specified in \stu{gadget's name}
  \vItem \stb{Scratch registers} => \stu{internal} registers for \stu{temporary storage}
  \vItem these \stu{scratch registers} are \stub[orange]{initially} \stu{set to zero}
  \& \stub[orange]{upon exit} \stu{must return to zero}
  \vItem Compose \stu{smaller gadgets} to create \stu{bigger gadgets} => can \stb{rename} clashing scratch register names
\end{enumerate}

\hSep

\kwub{Common RM gadgets:}

\mkSvg[101.5pt]{image2}
\^{} \stb{NOTE:} the \kwu{hollow arrow} coming out of \stu{gadgets} represents \textbf{all outputs} combined

\hSep

\stub{RM Gadgets for Coded Lists:}

foo bar

\hSep