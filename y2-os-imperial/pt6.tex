\mkImg{image copy 21}

\textbf{Race condition} occurs when multiple threads read and
write shared data within \ul{critical section} =>
their execution \ul{interleaved non-deterministically}.

\mkImg{image copy 22}

\hSep

\textbf{Synchronisation mechanism} required at entry/exit of
critical section => \ul{mediate} concurrent access \& prevent race conditions.

\mkImg{image copy 23}

\textbf{TEST \& SET LOCK} is an atomic instruction $\mathrm{TSL}(L)$
provided by most CPUs. Sets memory location $L$ to 1 and returns old value;
\ul{locks} using \ul{busy waiting} (e.g. \(\textcolor{cornellred}{\textrm{while (TSL($L$) != 0) $/* wait */$; }}\))
are \textbf{spin locks}.

\mkImg{image copy 24}

\textbf{STRICT ALTERNATION} fails to meet rules of mutual exclusion =>
if $T_0$'s non-crit. section is \ul{long enough}, it will prevent $T_1$
from entering crit. section. $\Downarrow \Downarrow \Downarrow$
\mkImg{image copy 26-pdf}

\textbf{PETERSON'S SOLUTION} fixes issues w/ strict alternation; thread $T_k$ calls
\(\textrm{enter/leave\_critical($k$)}\); $\Downarrow \Downarrow$
\mkImg{image copy 27-pdf}

\hSep

\textbf{Busy waiting} => synch. mechanism for which the
\ul{waiting thread} remains in \ul{running state} \textit{(e.g. spin locks)} which wastes CPU cycles,
and runs into \textbf{Priority Inversion Problem} => if \ul{higher-priority} $H$ is busy-waiting
on lock held by \ul{lower-priority} $L$, then $H$ is \ul{always scheduled} meaning $L$
cannot make progress, i.e. \ul{neither can make progress}.

\mkImg{image copy 25}
\^{} a commonly desired feature of \ul{synchronisation mechanisms}.

\textbf{Example implementations} => disabling interrupts, strict alternation, Peterson's solution,
spin locks, binary semaphores w/ init. $counter{=}1$.

\hSep

\textbf{LOCK/MUTEX} => \ul{mutual exclusion} mechanism \ul{owned} by \ul{actively excluding thread};
\textbf{lock} is usually \ul{user/process object} while
\textbf{mutex} is same thing but \ul{OS/Kernel object}.

\textbf{Lock granularity} => the amount of data a lock is protecting.

\textbf{Lock overhead} => \ul{cost} of using locks, e.g. memory space, initialization,
time required to acquire/release locks.

\textbf{Lock contention} =>  number of processes \ul{waiting} for lock; more contention => less parallelism.

\textbf{Read/Write Locks} can be used to \ul{minimize lock contention} \& \ul{maximize concurrency}  => exclusive access
in write mode but multiple threads can acquire in read mode.

\hSep

\mkImg{image copy 29}
\mkImg{image copy 28-pdf}

\hSep

\textbf{MONITORS on COND-VARS:}

\begin{enumerate}
    \vItem \textbf{Monitors} regulate access to shared data \& consist of (implicit) \textit{\textbf{monitor lock}}
    and $\geq 1$ \textit{\textbf{cond-vars}};
    \textit{\textbf{entry}} procedures called from outside the monitor, \textit{\textbf{internal}}
    procedures only callable from \ul{monitor procedures}, processes can only call entry procedures — only one
    process can be in the monitor at one time
    \vItem \textbf{Cond-vars} associated with high-level conditions; if \textit{\textbf{cond-var}}
    signalled w/ no one waiting for it then the signal is \ul{lost}; has operations:
    
    \begin{enumerate}
        \vItem \textcolor{cornellred}{\textit{wait(c)}} => 
        release monitor lock \& wait for $c$ to be signalled
        \vItem \textcolor{cornellred}{\textit{signal(c)}} => 
        wake up one process waiting for $c$
        \vItem \textcolor{cornellred}{\textit{broadcast(c)}} => 
        wake up all processes waiting for $c$
    \end{enumerate}

    \vItem \textbf{Monitors} are a \textit{language construct} and \textit{not supported by C}
    \textit{(supported by PintOS via \ul{explicit} monitor locks)}. Java supports \ul{variant} of monitors
    via $\mathbf{synchronized}$ methods, but they \ul{lack condition variables},
    i.e. no wait()/notify()
\end{enumerate}

\textbf{Producer/consumer example w/ monitors} $\Downarrow \Downarrow$
\mkImg[90pt]{image copy 30}

\^{} producer inserts \& consumer removes items w/ max. capacity $N$ => only insert/remove if mutual exclusion achieved
=> only insert if non-full \& remove if non-empty.

\hSep