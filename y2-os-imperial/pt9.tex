\subsection*{I/O Devices: Block, Character, Other}

I/O devices can be roughly divided into two categories:
\textbf{ \textit{block devices}} and \textbf{ \textit{character devices}}:

\begin{itemize}
    \vItem
          \textbf{ \textit{Block devices}}: store data in fixed-size blocks, each
          block one with its own address

          \begin{itemize}
              \vItem
                    Transfers in units of  \textit{one or more} entire (consecutive) blocks
              \vItem
                    Possible to read/write each block independently of the other blocks
              \vItem
                    e.g. HDDs, SSDs, magnetic tape drives, etc.
          \end{itemize}
    \vItem
          \textbf{ \textit{Character devices}}: delivers/accepts a stream of
          characters, disregarding any block structure

          \begin{itemize}

              \vItem
                    Not addressable and has no seek operation
              \vItem
                    e.g. printers, network interfaces, mice, etc.
          \end{itemize}
\end{itemize}

This classification scheme is not perfect. Some devices do not fit in:

\begin{itemize}

    \vItem
          \textbf{ \textit{Clocks}}:

          \begin{itemize}

              \vItem
                    not block addressable \& don't generate/accept character streams
              \vItem
                    They just cause interrupts at well-defined intervals
          \end{itemize}
    \vItem
          \textbf{ \textit{Memory-mapped screens}}: don't fit into this model
    \vItem
          \textbf{ \textit{Touch screens:}} don't fit into this model
\end{itemize}

\subsection*{Device Controllers/Adapters}

I/O units often consist of a mechanical component and an electronic
component. The electronic component is called the \textbf{ \textit{device
        controller}} or \textbf{ \textit{adapter}}.

\begin{itemize}

    \vItem
          On PCs it is often a chip on the motherboard or a
          \ul{PCB} which
          can be inserted into a (PCIe) expansion slot
    \vItem
          Controller usually has a connector on it, into which a cable leading
          to the device itself can be plugged
    \vItem
          Many controllers can handle multiple devices
    \vItem
          If the interface between the controller \& device is standard,
          companies make controllers/devices that fit that interface

          \begin{itemize}

              \vItem
                    e.g. SATA, SCSI, USB, etc.
          \end{itemize}
\end{itemize}

Each controller has a few registers that are used for communicating with
the CPU:

\begin{itemize}

    \vItem
          Writing into these registers: commands the device to deliver
          data/accept data/switch itself on or off/etc.
    \vItem
          Reading from these registers: learn what the device's state is,
          e.g. is is prepared to accept a new command, etc
\end{itemize}

Many \textbf{ \textit{devices}} have a data buffer that can be read+written
to:

\begin{itemize}

    \vItem
          e.g. a common way for computers to display pixels on the screen is
          video RAM

          \begin{itemize}

              \vItem
                    a data buffer for programs/OS to write into
          \end{itemize}
\end{itemize}

\subsection*{Memory-Mapped \& Port-Mapped I/O}

The \textbf{\textit{Port-Mapped I/O}} approach assigns an
\textbf{\textit{I/O port}} number to \ul{control registers and buffers}:

\begin{itemize}

    \vItem
          Set of all the \textbf{ \textit{I/O ports}} form the I/O port space
    \vItem
          Access to \textbf{ \textit{I/O ports}} is protected

          \begin{itemize}

              \vItem
                    ordinary user programs cannot access it
              \vItem
                    only the operating system can
          \end{itemize}
    \vItem
          Access is done with special I/O instructions,
          e.g. \texttt{IN\ REG,PORT} and \texttt{OUT\ PORT,REG}
    \vItem
          Most early computers worked this way
\end{itemize}

The \textbf{ \textit{Memory-Mapped I/O}} approach maps all \ul{control registers and buffers} to the
memory address space:

\begin{itemize}

    \vItem
          Commonly at/near the top of the address space
    \vItem
          \textbf{ \textit{Pros}}:

          \begin{itemize}

              \vItem
                    no need for special ASM instructions, so \ul{device drivers} can be written entirely in
                    C/C++
              \vItem
                    no special protection to memory-mapped addresses, so the OS can be
                    flexible with how it delegates device access

                    \begin{itemize}

                        \vItem
                              e.g. omit memory-mapped addresses from \ul{user's virtual address
                              space}
                        \vItem
                              e.g. give some users control over specific devices but not others,
                              allowing \ul{device drivers} to
                              run in User Mode

                              \begin{itemize}

                                  \vItem
                                        prevents a driver crash from taking down the entire system
                              \end{itemize}
                    \end{itemize}
              \vItem
                    reuse the same ASM instructions as those used on normal memory

                    \begin{itemize}

                        \vItem
                              one instruction can do two things, e.g. instead of first reading
                              an \textbf{ \textit{I/O port}} then doing something with it, you can
                              just do that thing in one go
                        \vItem
                              this reduces the number of instructions needed to be executed,
                              therefore more efficient
                    \end{itemize}
          \end{itemize}
    \vItem
          \textbf{ \textit{Cons}}:

          \begin{itemize}

              \vItem
                    Modern computers often \ul{cache
                    memory} but caching memory-mapped addresses is
                    \textbf{ \textit{DISASTEROUS}}

                    \begin{itemize}

                        \vItem
                              both hardware \& OS have to be designed to selectively disable
                              caching which can be very complex
                    \end{itemize}
              \vItem
                    Modern systems also have multiple buses (memory, PCIe, SCSI, and
                    USB)

                    \begin{itemize}

                        \vItem
                              I/O devices have no way of seeing memory addresses as they go by
                              on the memory bus
                        \vItem
                              Additional hardware/software complexity is needed to resolve this
                              issue

                              \begin{itemize}

                                  \vItem
                                        e.g. CPU uses non-memory buses as fallback if memory bus fails
                                        to respond
                                  \vItem
                                        e.g. put a snooping device on memory bus \& pass all addresses
                                        presented to potentially interested I/O devices
                                  \vItem
                                        e.g. preload address-ranges to memory-controller chip at boot
                                        time \& it will redirect addresses to the appropriate bus
                                        thereafter
                              \end{itemize}
                    \end{itemize}
          \end{itemize}
\end{itemize}

\subsection*{Direct Memory Access (DMA)}

The CPU can requesting data from an \ul{I/O controller} one byte at a time
wastes its time, so \textbf{ \textit{DMA (Direct Memory Access)}} is often
used:

\begin{itemize}

    \vItem
          Can only use \textbf{ \textit{DMA}} if hardware has a DMA controller,
          most systems do
    \vItem
          Most commonly its a single DMA controller (e.g. on the motherboard)

          \begin{itemize}

              \vItem
                    for regulating transfers to multiple devices,
              \vItem
                    often concurrently
          \end{itemize}
    \vItem
          DMA controller has access to system bus independent of the CPU
    \vItem
          Contains several registers that can be written/read by CPU

          \begin{itemize}

              \vItem
                    e.g. memory address register, byte count register, control
                    registers, etc.
              \vItem
                    these specify the I/O port to use, direction of transfer, transfer
                    unit, number of bytes to transfer in one burst, etc.
          \end{itemize}
\end{itemize}

The CPU programs the DMA controller and lets it run parallel to the CPUs
execution:

\begin{itemize}

    \vItem
          The DMA controller instructs the \ul{device I/O controllers} to perform
          the appropriate operations
    \vItem
          When one unit of work is done, the \ul{device I/O controllers} sends an
          acknowledgment signal to the DMA controller
    \vItem
          For each acknowledgment signal, the DMA controller decrements a
          counter which keeps track of progress
    \vItem
          When the counter reaches 0, the workload is complete and the DMA
          controller sends an appropriate \ul{interrupt} to the CPU
\end{itemize}

The above are simple cases of DMA controllers, but ones with more
complex capabilities exist:

\begin{itemize}

    \vItem
          Can be programmed to handle multiple transfers at the same time

          \begin{itemize}

              \vItem
                    multiple sets of control registers internally, one for each channel
              \vItem
                    may use round-robin or priority queue internally to decide which
                    device to service next
              \vItem
                    often each \ul{device controller} being serviced gets a separate channel for
                    acknowledgment signals
          \end{itemize}
    \vItem
          Buses can often operate in two modes: word-at-a-time \& block mode

          \begin{itemize}

              \vItem
                    Often, DMA controllers can also operate in either mode
              \vItem
                    In word-at-a-time mode: the DMA controller requests the transfer of
                    one word and gets it

                    \begin{itemize}

                        \vItem
                              If the CPU also wants the bus, it has to wait
                        \vItem
                              The mechanism is called cycle stealing
                    \end{itemize}
              \vItem
                    In block mode: DMA controller tells the \ul{device} to acquire the bus, issue
                    a series of transfers, then release the bus

                    \begin{itemize}

                        \vItem
                              this is called \textbf{ \textit{burst mode}}
                        \vItem
                              multiple words can be transferred for the price of one bus
                              acquisition
                        \vItem
                              can block the CPU and other devices for a substantial period if a
                              long burst is being transferred
                    \end{itemize}
          \end{itemize}
    \vItem
          In the above model (fly-by-mode), the the DMA controller tells the
          device controller to transfer the data directly to main memory

          \begin{itemize}

              \vItem
                    some DMA controllers have the \ul{device controller} send the word
                    to the DMA controller,

                    \begin{itemize}

                        \vItem
                              which then writes the word to wherever it is supposed to go
                    \end{itemize}
              \vItem
                    This scheme requires an extra bus cycle per word transferred,

                    \begin{itemize}

                        \vItem
                              but is more flexible in that it can also perform device-to-device
                              copies and even memory-to-memory copies
                    \end{itemize}
          \end{itemize}
\end{itemize}

\subsection*{Interrupt handlers for hardware interrupts}

\begin{itemize}

    \vItem
          For \ul{block devices}: on transfer completion, signal \ul{device handler}
    \vItem
          For \ul{character devices}: when character transferred, process next character
\end{itemize}
\mkImg{pastedimg2}

\subsection*{Goals of I/O Software}

\textbf{ \textit{Device independence}}: write programs that can access any
I/O device without having to specify the device type or device instance
in advance

\begin{itemize}

    \vItem
          Some differences which the OS has to abstract are:

          \begin{itemize}

              \vItem
                    Unit of data transfer: character or block
              \vItem
                    Supported operations: e.g. read, write, seek
              \vItem
                    Synchronous or asynchronous operation
              \vItem
                    Speed differences
              \vItem
                    Sharable (e.g. disks) or single user (e.g. printer, DVD-RW)
              \vItem
                    Types of error conditions
    \end{itemize}
\end{itemize}

\textbf{\textit{Uniform naming}}: name of
                    file/device should be a string/integer not dependent on device in
                    any way
                    
                    
\textbf{ \textit{Error handling}}: should be handled as close to the hardware as possible
\begin{enumerate}
    \vItem
    Either the controller or \ul{driver}
    should handle the error transparently
\vItem
    Only if the lower layers are not able to deal with the problem should
    the upper layers be told about it 
\end{enumerate}
          
\textbf{ \textit{Synchronous (blocking)
                  vs. asynchronous (interrupt-driven)}}:
\begin{enumerate}
    \vItem
    Most physical I/O is asynchronous
\vItem
    Most I/O code is easier to write synchronously
\vItem
    The OS must make interrupt-driven operations appear blocking to the
    user programs
\vItem
    The OS must  \textit{still} provide asynchronous API for use in
    performance-critical user programs 
\end{enumerate}
          
          
\textbf{ \textit{Buffered
                  vs. non-buffered I/O}}: often data that come off a device cannot be
          stored directly in their final destination
\begin{enumerate}
    \vItem
          e.g. packets coming off a network cannot be directed to the right
          place until they have been stored and inspected for their port number
    \vItem
          some destination devices (e.g. digital audio devices) have limitations
          on data-transfer rates, so a buffer is used to ``normalize'' those
    \vItem
          Buffering involves considerable copying and often has a major impact
          on I/O performance
\end{enumerate}

\mkImg{image copy 46}

\subsection*{Ways to do I/O}

\begin{enumerate}
    \def\labelenumi{\arabic{enumi}.}

    \vItem
          \textbf{ \textit{Programmed I/O}}: CPU programs \ul{I/O control registers} directly and
          blocks until the operation is done

          \begin{itemize}

              \vItem
                    This is done via \textbf{ \textit{polling}} the \ul{device} until its done, called
                    \textbf{ \textit{busy waiting}}
              \vItem
                    Wastes CPU time and inefficient
          \end{itemize}
    \vItem
          \textbf{ \textit{Interrupt-Driven I/O}}: The CPU programs
          \ul{I/O control registers} directly and continues doing other things

          \begin{itemize}

              \vItem
                    the \ul{device} will
                    send an \ul{interrupt} to the CPU
                    once done
              \vItem
                    can still be very inefficient with unbuffered \ul{character devices} which will
                    interrupt on every character
          \end{itemize}
    \vItem
          \textbf{ \textit{I/O using Direct Memory Access (DMA)}}: The CPU programs
          \ul{a DMA controller} and
          continues doing other things

          \begin{itemize}

              \vItem
                    the DMA controller will perform the entire operation and send an
                    \ul{interrupt} to the CPU once done
              \vItem
                    may be inefficient in certain circumstances, because \ul{DMA controllers} are usually much
                    slower than the CPU
          \end{itemize}
\end{enumerate}

\section*{I/O Software Layers}

I/O software is typically organized in four layers. Each layer has a
well-defined function to perform and a well-defined interface to the
adjacent layers
\mkImg{Pasted image 20250106201355}

\subsection*{Interrupt-Handlers in the Software Layers}

\ul{Interrupt handlers} should aim to be as buried as possible, doing as little
as possible.

\begin{itemize}

    \vItem
          e.g. have a \ul{driver} start an I/O
          operation and block (like by \ul{downing a semaphore}) and the
          interrupt-handler only unblocks it (like by upping that semaphore)
    \vItem
          e.g. for others its a \ul{condvar signal} or other kind of message etc. which will make the device
          driver resume
    \vItem
          the above works best if the driver is structured as a \ul{process}
\end{itemize}

\subsection*{Device Drivers}

Handles one device type, but many devices of that type. 
Its functions are implementing block read/write, accessing
device registers, initiating operations, scheduling requests,
handling errors.

Most operating systems define a \ul{standard interface} that all block drivers must support and a
\ul{second standard interface} that all character drivers must support.


\subsection*{Device-Independent I/O software}

The basic function of the \textbf{ \textit{device-independent software}} is
to perform the I/O functions that are common to all devices and to
provide a uniform interface to the user-level software =>

uniform interfacing for device drivers => buffering => error reporting =>
allocating and releasing dedicated devices => providing a device-independent block size

\subsubsection*{Uniform Interfacing for Device Drivers}

A major issue in an OS is how to make all I/O devices and drivers look
more or less the same. If they weren't the same, we would have to hack
on the OS for each new device

Not all devices are absolutely
identical, but there are only a small number of device types which need
to be accounted for

\begin{itemize}

    \vItem
          e.g. classes of device like disks, printers, network stream
    \vItem
          OS defines a set of functions that the driver of a specific device
          type must supply The device-independent software takes care of mapping
          symbolic device names onto the proper driver:
    \vItem
          e.g. for UNIX a device name, such as /dev/disk0, uniquely specifies
          the i-node for a special file

          \begin{itemize}

              \vItem
                    this i-node contains the major device number, which is used to
                    locate the appropriate driver
              \vItem
                    The i-node also contains the minor device number, which is passed as
                    a parameter to the driver in order to specify the unit to be read or
                    written
          \end{itemize}
\end{itemize}

\subsubsection*{Buffering}

\textbf{Buffered I/O} — User data transferred to OS output
buffer. Process continues, only suspends when buffer full.
OS reads ahead but process blocks when buffer empty.

\textbf{Unbuffered I/O} — Data transferred directly to/from user
space from/to device. Device handler used for each
transfer as each read/write is causing physical I/O. High
process switching overhead.

\textbf{I/O operations} — open, close, read, write, seek.


\subsubsection*{Device-Independent Block Size}

Different SSDs have different flash page sizes, while different disks
may have different sector sizes. It is up to the device-independent
software to hide this fact and provide a uniform block size to higher
layers

\subsection*{User-Space I/O Software}

\subsubsection*{User-Space I/O Libraries}

Most user-space I/O code is library code which simply arranges arguments
in the right way for a SYSCALL,

\begin{itemize}

    \vItem
          sometimes more complex logic is done i.e. \texttt{printf} and
          \texttt{scanf} in C
\end{itemize}

\subsubsection*{Spooling Daemons}

\mkImg{Pasted image 20250106205848}

\subsection*{Device management on Linux}
\mkImg{image copy 47}
\mkImg{image copy 48}
\mkImg{image copy 49}

\hSep