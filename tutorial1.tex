\subsection*{Basic identities of matrix/vector ops}

\iMbox{(A+B)^{T} = A^{T} + B^{T}}  \iMbox{(AB)^{T} = B^{T}A^{T}}
\iMbox{(A^{-1})^{T}=(A^{T})^{-1}} \iMbox{(AB)^{-1} = B^{-1}A^{-1}}

\hSep % --- Next section ---

For \iMbox{A \in \mathbb{R}^{m \times n}}, \iMbox{A_{ij}} is the \iMbox{i}-th \textbf{ROW} then \iMbox{j}-th \textbf{COLUMN} \\
\iMbox{(A^T)_{ij} = A_{ji}}
\iMbox{\ds (AB)_{ij} = A_{i\ast} \cdot B_{\ast j} = \sum_{k} A_{ik}B_{kj}}
\iMbox{(Ax)_{i} = A_{i\ast} \cdot x = \sum_{j} A_{ij} x_{j}}
\iMbox{x^Ty = y^Tx = x \cdot y = \sum_{i} x_{i} y_{i}}
\iMbox{x^TAx = \sum_{i} \sum_{j} A_{ij}x_{i}x_{j}}
\iMbox{\mathbfit{x} \mathbf{e}_{k}^{T} = [\mathbf{0}|\dots | \mathbfit{x} | \dots |\mathbf{0}]}
\iMbox{\mathbf{e}_{k} \mathbfit{x}^{T} = [\mathbf{0}^{T}; \dots ; \mathbfit{x}^{T} ; \dots ; \mathbf{0}^{T}]}

\hSep % --- Next section ---

Scalar-multiplication + addition distributes over:
\begin{itemize}
      \vItem
      \textbf{column-blocks} =>
      \iMbox{\lambda A+B = \lambda [A_{1}|\dots|A_{c}] + [B_{1}|\dots|B_{c}] = [\lambda A_{1}+B_{1}|\dots|\lambda A_{c}+B_{c}]}
      \vItem
      \textbf{row-blocks} =>
      \iMbox{\lambda A+B = \lambda [A_{1}; \dots; A_{r}] + [B_{1};\dots; B_{r}] = [\lambda A_{1}+B_{1}; \dots; \lambda A_{r}+B_{r}]}
\end{itemize}

Matrix-multiplication distributes over:
\begin{itemize}
      \vItem
      \textbf{column-blocks} =>
      \iMbox{\ds AB = A[B_{1}|\dots|B_{p}] = [AB_{1}|\dots|AB_{p}]}
      \vItem
      \textbf{row-blocks} =>
      \iMbox{AB = [A_{1};\dots;A_{p}]B = [A_{1}B;\dots;A_{p}B]}
      \vItem
      \textbf{\emph{outer-product sum}} =>
      \iMbox{AB = [A_{1}|\dots|A_{p}][B_{1};\dots; B_{p}] = \sum_{i=1}^{p} A_{i}B_{i}}
      \begin{itemize}
            \vItem
            e.g. for \iMbox{A = [\mathbf{a}_{1}|\dots|\mathbf{a}_{n}]},
            \iMbox{B=[\mathbf{b}_{1}; \dots; \mathbf{b}_{n}]} =>
            \iMbox{AB = \sum_{i} \mathbf{a}_{i}\mathbf{b}_{i}}
      \end{itemize}
\end{itemize}

\subsection*{Projection: definition \& properties}

\begin{itemize}
      \vItem
            A \textbf{projection} \iMbox{\pi: V \to V} is a \textbf{endomorphism}
            such that \iMbox{\ds \pi \circ \pi = \pi}, i.e.~it leaves its image
            unchanged (its idempotent)
      \vItem
            A \textbf{square matrix} \iMbox{P} such that \iMbox{P^2 = P} is called
            a \textbf{projection matrix}

            \begin{itemize}

                  \vItem
                        It is called an \textbf{orthogonal projection matrix} if
                        \iMbox{P^2 = P = P^{\dagger}} (conjugate-transpose)
                  \vItem
                        Eigenvalues of a \textbf{projection matrix} must be 0 or 1
            \end{itemize}
      \vItem
            Because \iMbox{\pi: V \to V} is a \textbf{linear map}, its
            \textbf{image space} \iMbox{U = \mathrm{im}(\pi)} and \textbf{null
                  space} \iMbox{W = \mathrm{ker}(\pi)} are \textbf{subspaces} of
            \iMbox{V}

            \begin{itemize}

                  \vItem
                        \iMbox{\pi} is the \textbf{identity operator} on \iMbox{U}
                  \vItem
                        The \textbf{linear map} \iMbox{\pi^{*} = \mathrm{I}_{V} - \pi} is
                        \textbf{also} a projection with
                        \iMbox{W = \mathrm{im}(\pi^{*}) = \mathrm{ker}(\pi)} and
                        \iMbox{U = \mathrm{ker}(\pi^{*}) = \mathrm{im}(\pi)}, i.e.~they
                        swapped

                        \begin{itemize}

                              \vItem
                                    \iMbox{\pi} is a projection \textbf{along} \iMbox{W} \textbf{onto}
                                    \iMbox{U}
                              \vItem
                                    \iMbox{\pi^{*}} is a projection \textbf{along} \iMbox{U}
                                    \textbf{onto} \iMbox{W}
                              \vItem
                                    \iMbox{\pi^{*}} is the \textbf{identity operator} on \iMbox{W}
                        \end{itemize}
                  \vItem
                        \iMbox{V} can be decomposed as \iMbox{V = U \oplus W} meaning every
                        vector \iMbox{x \in V} can be uniquely written as \iMbox{x = u + w}

                        \begin{itemize}

                              \vItem
                                    \iMbox{u \in U} and \iMbox{u = \pi(x)}
                              \vItem
                                    \iMbox{w \in W} and
                                    \iMbox{w = x - \pi(x) = (I_{V}-\pi )(x) = \pi^{*}(x)}
                        \end{itemize}
            \end{itemize}
      \vItem
            An \textbf{orthogonal projection} further satisfies \iMbox{U \bot W}
            i.e.~the \textbf{image} and \textbf{kernel} of \iMbox{\pi} are
            \textbf{orthogonal subspaces}

            \begin{itemize}

                  \vItem
                        infact they are eachother's \textbf{orthogonal compliments},
                        i.e.~\iMbox{U^{\perp} = W, W^{\perp} = U} \emph{(because
                              finite-dimensional vectorspaces)}
                  \vItem
                        so we have
                        \iMbox{\pi(x) \cdot y = \pi(x) \cdot \pi(y) = x \cdot \pi(y)}
                  \vItem
                        or equivalently,
                        \iMbox{\pi(x) \cdot (y - \pi(y)) = (x - \pi(x)) \cdot \pi(y) = 0}
            \end{itemize}
\end{itemize}

\hSep % --- Next section ---

\begin{itemize}

      \vItem
            By
            Cauchy--Schwarz inequality we have \iMbox{\lVert \pi(x) \rVert \leq \lVert x \rVert}
      \vItem
            The \textbf{orthogonal projection onto the line} containing vector
            \iMbox{u} is \iMbox{\ds \mathrm{proj}_{u} = \hat{u} \hat{u}^{T}},
            i.e. \iMbox{\ds\mathrm{proj}_{u}(v) = \frac{u \cdot v}{u \cdot u} u;
                  \hat{u} = \frac{u}{\lVert u \rVert}}

            \begin{itemize}
                  \vItem
                        A special case of \iMbox{\pi(x) \cdot (y - \pi(y)) = 0} is
                        \iMbox{u \cdot(v - \mathrm{proj}_{u}v) = 0}, since
                        \iMbox{\ds\mathrm{proj}_{u}(u) = u}
            \end{itemize}
      \vItem
            If \iMbox{U \subseteq \mathbb{R}^{n}} is a \iMbox{k}-dimensional
            subspace with \textbf{orthonormal basis (ONB)}
            \iMbox{\langle \mathbf{u}_{1},\dots,\mathbf{u}_{k} \rangle \in \mathbb{R}^m}

            \begin{itemize}

                  \vItem
                        Let
                        \iMbox{\ds \mathbf{U} = [ \mathbf{u}_{1}|\dots|\mathbf{u}_{k} ] \in \mathbb{R}^{m \times k}} matrix
                  \vItem
                        \textbf{Orthogonal projection onto} \iMbox{U} is \iMbox{\ds \pi_{U} = \mathbf{U} \mathbf{U}^{T}}
                  \vItem
                        Can be rewritten as
                        \iMbox{\ds\pi_{U}(v) = \sum_{i} (\mathbf{u}_{i} \cdot v) \mathbf{u}_{i}}
                  \vItem
                        If \iMbox{\langle \mathbf{u}_{1},\dots,\mathbf{u}_{k} \rangle} is
                        \textbf{not orthonormal}, then ``normalizing factor''
                        \iMbox{(\mathbf{U}^{T}\mathbf{U})^{-1}} is added =>
                        \iMbox{\ds \pi_{U} = \mathbf{U}(\mathbf{U}^{T}\mathbf{U})^{-1}\mathbf{U}^{T}}

                        \begin{itemize}

                              \vItem
                                    For \textbf{line subspaces} \iMbox{U = \mathrm{span} \{ u \}}, we
                                    have
                                    \iMbox{\ds(\mathbf{U}^{T}\mathbf{U})^{-1} = \ds(u^{T}u)^{-1} = 1 / (u \cdot u) = 1 / \lVert u \rVert}
                        \end{itemize}
            \end{itemize}
\end{itemize}

\subsection*{Gram-Schmidt (GS) to gen. ONB from lin. ind. vectors}
\begin{itemize}

      \vItem
            Gram-Schmidt is \textbf{iterative} projection => we use
            \textbf{\emph{current} \iMbox{j}-dim subspace}, to get
            \textbf{\emph{next} \iMbox{(j+1)}-dim subspace}

            \begin{itemize}

                  \vItem
                        Assume \textbf{orthonormal basis (ONB)}
                        \iMbox{\ds\langle \mathbf{q}_{1},\dots,\mathbf{q}_{j} \rangle \in \mathbb{R}^m}
                        for \textbf{\iMbox{j}-dim subspace
                              \iMbox{\ds U_{j} \subset \mathbb{R}^{m}}}

                        \begin{itemize}

                              \vItem
                                    Let
                                    \iMbox{\ds Q_{j} = [ \mathbf{q}_{1}|\dots|\mathbf{q}_{j} ] \in \mathbb{R}^{m \times j}}
                                    be the matrix
                              \vItem
                                    \iMbox{\ds \mathrm{P}_{j} = Q_{j}Q_{j}^{T}} is orthogonal projection \textbf{onto
                                          \iMbox{\ds U_{j}}}
                              \vItem
                                    \iMbox{\ds \mathrm{P}_{\perp j} = \mathbf{I}_{m} - Q_{j}Q_{j}^{T}}
                                    is orthogonal projection \textbf{onto
                                          \iMbox{\ds \left(U_{j}\right)^{\perp}}} \emph{(orthogonal
                                          compliment)}
                        \end{itemize}
                  \vItem
                        Uniquely decompose next \iMbox{\ds U_{j} \not\ni \mathbf{a}_{j+1} = \mathbf{v}_{j+1} + \mathbf{u}_{j+1}}
                        \begin{itemize}

                              \vItem
                                    \iMbox{\mathbf{v}_{j+1} = \mathrm{P}_{j}\left(\mathbf{a}_{j+1}\right) \in U_{j}}
                                    => discard it!!
                              \vItem
                                    \iMbox{\mathbf{u}_{j+1} = \mathrm{P}_{\perp j}\left(\mathbf{a}_{j+1}\right) \in \left(U_{j}\right)^{\perp}}
                                    => we're after this!!
                        \end{itemize}
                  \vItem
                        Let \iMbox{\ds\mathbf{q}_{j+1} = \widehat{\mathbf{u}}_{j+1}}
                        => we have \textbf{\emph{next ONB}}
                        \iMbox{\ds\langle \mathbf{q}_{1},\dots,\mathbf{q}_{j+1} \rangle} for
                        \iMbox{\ds U_{j+1}} => start next iteration

                        \begin{itemize}

                              \vItem
                                    \iMbox{\ds \mathbf{u}_{j+1} = \left(\mathbf{I}_{m} - Q_{j}Q_{j}^{T} \right)\mathbf{a}_{j+1} = \mathbf{a}_{j+1} - Q_{j}\mathbf{c}_{j}}
                                    where
                                    \iMbox{\ds \mathbf{c}_{j} = [\mathbf{q}_{1} \cdot \mathbf{a}_{j+1}, \dots, \mathbf{q}_{j} \cdot \mathbf{a}_{j+1}]^{T}}
                                    \tcbbreak
                              \vItem
                                    \textbf{Notice:}
                                    \iMbox{\ds  Q_{j}\mathbf{c}_{j} = \sum_{i=1}^{j} (\mathbf{q}_{i} \cdot \mathbf{a}_{j+1})\mathbf{q}_{i} = \sum_{i=1}^{j} \mathrm{proj}_{\ds \mathbf{q}_{i}}(\mathbf{a}_{j+1})},
                                    so rewrite as
                                    \iMbox{\ds \mathbf{u}_{j+1} = \mathbf{a}_{j+1} - \sum_{i=1}^{j} (\mathbf{q}_{i} \cdot \mathbf{a}_{j+1})\mathbf{q}_{i} = \mathbf{a}_{j+1} - \sum_{i=1}^{j} \mathrm{proj}_{\ds \mathbf{q}_{i}}(\mathbf{a}_{j+1})}
                        \end{itemize}
            \end{itemize}
      \vItem
            Let \iMbox{\ds\mathbf{a}_{1},\dots,\mathbf{a}_{n} \in \mathbb{R}^m}
            \emph{(\iMbox{m \geq n})} be linearly independent, i.e. basis of
            \textbf{\iMbox{n}-dim subspace}
            \iMbox{\ds U_{n} = \mathrm{span} \{ \mathbf{a}_{1},\dots,\mathbf{a}_{n}  \}}

            \begin{itemize}

                  \vItem
                        We apply Gram-Schmidt to build \textbf{ONB}
                        \iMbox{\ds\langle \mathbf{q}_{1},\dots,\mathbf{q}_{n} \rangle \in \mathbb{R}^m}
                        for \iMbox{\ds U_{n} \subset \mathbb{R}^{m}}
                  \vItem
                        \iMbox{j=1} =>
                        \iMbox{\ds\mathbf{u}_{1} = \mathbf{a}_{1}} and
                        \iMbox{\ds\mathbf{q}_{1} = \widehat{\mathbf{u}}_{1}},
                        i.e.~\textbf{\emph{start of iteration}}
                  \vItem
                        \iMbox{j=2} =>
                        \iMbox{\ds\mathbf{u}_{2} = \mathbf{a}_{2} - (\mathbf{q}_{1} \cdot \mathbf{a}_{2})\mathbf{q}_{1}}
                        and \iMbox{\ds\mathbf{q}_{2} = \widehat{\mathbf{u}}_{2}},
                        \textbf{\emph{etc\ldots{}}}
                  \vItem
                        Linear independence \textbf{\emph{guarantees}} that
                        \iMbox{\ds \mathbf{a}_{j+1} \not\in U_{j}}
                  \vItem
                        \textbf{For exams}: compute \iMbox{\ds \mathbf{u}_{j+1} = \mathbf{a}_{j+1} - Q_{j}\mathbf{c}_{j}}

                        \begin{enumerate}
                              \vItem
                                    Gather
                                    \iMbox{\ds Q_{j} = [\mathbf{q}_{1}|\dots|\mathbf{q}_{j}] \in \mathbb{R}^{m \times j}}
                              \vItem
                                    Compute
                                    \iMbox{\ds \mathbf{c}_{j} = [\mathbf{q}_{1} \cdot \mathbf{a}_{j+1}, \dots, \mathbf{q}_{j} \cdot \mathbf{a}_{j+1}]^{T} \in \mathbb{R}^{j}}
                              \vItem
                                    Compute \iMbox{\ds Q_{j}\mathbf{c}_{j} \in \mathbb{R}^{m}}, and
                                    subtract from \iMbox{\mathbf{a}_{j+1}}
                        \end{enumerate}
            \end{itemize}
\end{itemize}

\subsection*{Properties: dot-product \& norm}


\iMbox{\ds x^Ty = y^Tx = x \cdot y = \sum_{i} x_{i} y_{i}}
\iMbox{x \cdot y = \lVert a \rVert \lVert b \rVert \cos \widehat{xy}}
\iMbox{x \cdot y = y \cdot x}
\iMbox{x \cdot (y + z) = x \cdot y + x \cdot z}
\iMbox{\alpha x \cdot y = \alpha(x \cdot y)}
\iMbox{x \cdot x = \lVert x \rVert^{2}= 0 \iff x = \mathbf{0}}

for \iMbox{x \neq \mathbf{0}}, we have
\iMbox{x \cdot y = x \cdot z \implies x \cdot (y-z) = 0}

\iMbox{\lvert x \cdot y \rvert \leq \lVert x \rVert \lVert y \rVert}
(\textbf{Cauchy-Schwartz inequality})

\iMbox{\ds\lVert u + v \rVert^{2} + \lVert u - v \rVert^{2} = 2 \lVert u \rVert^{2} + 2 \lVert v \rVert^{2}}
(\textbf{parallelogram law})

\iMbox{\lVert u+v \rVert \leq \lVert u \rVert + \lVert v \rVert}
(\textbf{triangle inequality})

\iMbox{u \perp v \iff \lVert u + v \rVert^{2} = \lVert u \rVert^{2} + \lVert v \rVert^{2}}
(\textbf{pythagorean theorem})

\iMbox{\lVert c \rVert^2 = \lVert a \rVert^2 + \lVert b \rVert^2 - 2\lVert a \rVert \lVert b \rVert \cos \widehat {ba}}
(\textbf{law of cosines})

\subsection*{Transformation matrix \& linear maps}


For linear map \iMbox{f: \mathbb{R}^n \to \mathbb{R}^m}, ordered bases
\iMbox{\langle \mathbf{b}_{1},\dots,\mathbf{b}_{n} \rangle \in \mathbb{R}^n}
and
\iMbox{\langle \mathbf{c}_{1},\dots,\mathbf{c}_{m} \rangle \in \mathbb{R}^m}

\begin{itemize}

      \vItem
            \iMbox{\ds A = \mathbf{F}_{CB} \in \mathbb{R}^{m \times n}} is the
            \textbf{transformation-matrix} of \iMbox{f} w.r.t to bases \iMbox{B}
            and \iMbox{C}
      \vItem
            \iMbox{f(\mathbf{b}_{j}) = \sum_{i=1}^{m} A_{ij} \mathbf{c}_{i}}
            --> each \iMbox{\mathbf{b}_{j}} basis gets mapped to a
            linear combination of \iMbox{\sum_{i} a_{i}\mathbf{c}_{i}} bases
      \vItem
            If \iMbox{f^{-1}} exists \emph{(i.e.~its bijective and \iMbox{m=n})}
            then \iMbox{(\mathbf{F}_{CB})^{-1} = {\mathbf{F}^{-1}}_{BC}}
            \emph{(where \iMbox{{\mathbf{F}^{-1}}_{BC}} is the
                  \textbf{transformation-matrix} of \iMbox{f^{-1}})}
\end{itemize}

\hSep % --- Next section ---

The transformation matrix of the identity map is called
change-in-basis matrix

\begin{itemize}

      \vItem
            The identity matrix \iMbox{\mathbf{I}_{m}} represents
            \iMbox{\ds \mathrm{id}_{\ds\mathbb{R}^{m}}} w.r.t. the standard
            basis
            \iMbox{E_{m} = \langle \mathbf{e}_{1},\dots,\mathbf{e}_{m} \rangle}
            => i.e.~\iMbox{\mathbf{I}_{m} = \mathbf{I}_{EE}}
      \vItem
            If \iMbox{B = \langle \mathbf{b}_{1},\dots,\mathbf{b}_{m} \rangle}
            is a basis of \iMbox{\mathbb{R}^m}, then
            \iMbox{\mathbf{I}_{EB} = [\mathbf{b}_{1}|\dots|\mathbf{b}_{m}]} is
            the transformation matrix from \iMbox{B} to \iMbox{E}
      \vItem
            \iMbox{\mathbf{I}_{BE} = (\mathbf{I}_{EB})^{-1}}, so =>
            \iMbox{\mathbf{F}_{CB} = \mathbf{I}_{CE}\mathbf{F}_{EE}\mathbf{I}_{EB}}
\end{itemize}

\hSep % --- Next section ---

Dot-product uniquely determines a vector w.r.t. to basis
\begin{itemize}

      \vItem
            If \iMbox{a_{i} = x \cdot \mathbf{b}_{i};
                  x = \sum_{i} a_{i} \mathbf{b}_{i}}, we call \iMbox{a} the
            coordinate-vector of \iMbox{x} w.r.t. to \iMbox{B}
\end{itemize}

\textbf{Rank-nullity theorem}:
\iMbox{\operatorname{dim}(\operatorname{im}(f))+\operatorname{dim}(\operatorname{ker}(f))=\operatorname{rk}(A)+\operatorname{dim}(\operatorname{ker}(A))=n}

\iMbox{f} is injective/monomorphism \textbf{iff}
\iMbox{\ker(f) = \{ \mathbf{0} \}} \textbf{iff} \iMbox{A} is full-rank

\subsection*{Orthogonality concepts}

\begin{itemize}

      \vItem
            \iMbox{u \perp v \iff u \cdot v = 0}, i.e.~\iMbox{u} and \iMbox{v} are
            orthogonal
      \vItem
            \iMbox{u} and \iMbox{v} are orthonormal \textbf{iff}
            \iMbox{u \perp v, \lVert u \rVert = 1 = \lVert v \rVert}
      \vItem
            \iMbox{A \in \mathbb{R}^{n \times n}} is orthogonal \textbf{iff}
            \iMbox{A^{-1}=A^{T}}

            \begin{itemize}

                  \vItem
                        Columns of \iMbox{A = [\mathbf{a}_{1}|\dots|\mathbf{a}_{n}]} are
                        orthonormal basis (ONB)
                        \iMbox{C = \langle \mathbf{a}_{1},\dots,\mathbf{a}_{n} \rangle \in \mathbb{R}^n},
                        so \iMbox{A = \mathbf{I}_{EC}} is change-in-basis matrix
                  \vItem
                        Orthogonal transformations preserve lengths/angles/distances
                        =>
                        \iMbox{\lVert Ax \rVert_{2} = \lVert x \rVert_{2}, \ \widehat{{Ax} {Ay}} = \widehat{xy}}

                        \begin{itemize}

                              \vItem
                                    Therefore can be seen as a succession of reflections and planar
                                    rotations
                        \end{itemize}
                  \vItem
                        \iMbox{\det(A) = 1} or \iMbox{\det(A) = -1}, and all
                        \textbf{eigenvalues} of \iMbox{A} are s.t.
                        \iMbox{\lvert \lambda \rvert = 1}
            \end{itemize}
      \vItem
            \iMbox{A \in \mathbb{R}^{m \times n}} is semi-orthogonal \textbf{iff}
            \iMbox{A^T A = I \ \ \text{or} \ \ AA^{T} = I}

            \begin{itemize}

                  \vItem
                        If \iMbox{n > m} then \textbf{all \iMbox{m} rows} are orthonormal
                        vectors
                  \vItem
                        If \iMbox{m > n} then \textbf{all \iMbox{n} columns} are orthonormal
                        vectors
            \end{itemize}
      \vItem
            \iMbox{U \perp V \subset \mathbb{R}^n \iff \mathbf{u} \cdot \mathbf{v} = 0}
            for all \iMbox{\mathbf{u}\in U,\mathbf{v}\in V}, i.e.~they are
            \textbf{orthogonal subspaces}
      \vItem
            \textbf{Orthogonal compliment} of \iMbox{U \subset \mathbb{R}^{n}} is
            the subspace
            \iMbox{\ds \begin{aligned}
                        U^{\perp} & = \left\{ \ x \in \mathbb{R} ^{n} \ \middle| \ \forall y\in \mathbb{R}^{n}:x \perp y \ \right\}                                \\
                                  & = \left\{ \ x \in \mathbb{R} ^{n} \ \middle| \ \forall y\in \mathbb{R}^{n}: \lVert x \rVert \leq \lVert x +y \rVert \ \right\}
                  \end{aligned}}

            \begin{itemize}

                  \vItem
                        \iMbox{\mathbb{R}^{n} = U \oplus U^{\perp}} and
                        \iMbox{(U^{\perp})^{\perp} = U}
                  \vItem
                        \iMbox{U \perp V \iff U^{\perp} = V} and vice-versa
                  \vItem
                        \iMbox{Y \subseteq X \implies X^{\perp} \subseteq Y^{\perp}} and
                        \iMbox{X \cap X^{\perp} = \{ \mathbf{0} \}}
                  \vItem
                        Any \iMbox{\mathbf{x} \in \mathbb{R}^{n}} can be uniquely decomposed
                        into \iMbox{\mathbf{x} = \mathbf{x}_{i} + \mathbf{x}_{k}}, where
                        \iMbox{\mathbf{x}_{i} \in U} and
                        \iMbox{\mathbf{x}_{k} \in U^{\perp}}
            \end{itemize}
      \vItem
            For matrix \iMbox{A \in  \mathbb{R}^{m \times n}} and for row-space
            \iMbox{\mathrm{R}(A)}, column-space \iMbox{\mathrm{C}(A)} and null
            space \iMbox{\mathrm{ker}(A)}

            \begin{itemize}

                  \vItem
                        \iMbox{\ds\mathrm{R}(A)^{\perp} = \mathrm{ker}(A)} and
                        \iMbox{\ds\mathrm{C}(A)^{\perp} = \mathrm{ker}(A^{T})}
                  \vItem
                        Any \iMbox{\mathbf{b} \in \mathbb{R}^{m}} can be uniquely decomposed
                        into

                        \begin{itemize}

                              \vItem
                                    \iMbox{\mathbf{b} = \mathbf{b}_{i} + \mathbf{b}_{k}}, where
                                    \iMbox{\mathbf{b}_{i} \in \mathrm{C}(A)} and
                                    \iMbox{\mathbf{b}_{k} \in \mathrm{ker}(A^{T})}
                              \vItem
                                    \iMbox{\mathbf{b} = \mathbf{b}_{i} + \mathbf{b}_{k}}, where
                                    \iMbox{\mathbf{b}_{i} \in \mathrm{R}(A)} and
                                    \iMbox{\mathbf{b}_{k} \in \mathrm{ker}(A)}
                        \end{itemize}
            \end{itemize}
\end{itemize}

\subsection*{Vector norms (beyond euclidean)}

\begin{itemize}

      \vItem
            \textbf{vector norms} are such that:
            \iMbox{\lVert x \rVert = 0 \iff x = \mathbf{0}},
            \iMbox{|\lambda x|=|\lambda| \lVert x \rVert},
            \iMbox{\|x+y\| \leq\|x\|+\|y\|}
      \vItem
            \iMbox{\ell_{p}} norms:
            \iMbox{\|\mathbf{x}\|_p=\left(\sum_{i=1}^n\left|\mathbf{x}_i\right|^p\right)^{1 / p}}

            \begin{itemize}

                  \vItem
                        \iMbox{p=1}:
                        \iMbox{\|\mathbf{x}\|_1=\sum_{i=1}^n\left|\mathbf{x}_i\right|}
                  \vItem
                        \iMbox{p=2}:
                        \iMbox{\|\mathbf{x}\|_2=\sqrt{\sum_{i=1}^n \mathbf{x}_i^2} = \sqrt{ \mathbf{x} \cdot \mathbf{x} }}
                  \vItem
                        \iMbox{p=\infty}:
                        \iMbox{\|\mathbf{x}\|_{\infty} = \lim _{p \rightarrow \infty}\|\mathbf{x}\|_p =\max _{1 \leq i \leq n}\left|\mathbf{x}_i\right|}
            \end{itemize}
      \vItem
            Any two norms in \iMbox{\mathbb{R}^n} are equivalent, meaning there
            exist \iMbox{r>0;s>0} such that:
            \iMbox{\forall \mathbf{x} \in \mathbb{R}^n, r\|\mathbf{x}\|_a \leq\|\mathbf{x}\|_b \leq s\|\mathbf{x}\|_a}

            \begin{itemize}

                  \vItem
                        Equivalence of \iMbox{\ell_1, \ell_2} and \iMbox{\ell_{\infty}}
                        =>
                        \iMbox{\begin{aligned} & \|\mathbf{x}\|_{\infty} \leq\|\mathbf{x}\|_2 \leq\|\mathbf{x}\|_1 \\ & \|\mathbf{x}\|_2 \leq \sqrt{n}\|\mathbf{x}\|_{\infty} \\ & \|\mathbf{x}\|_1 \leq \sqrt{n}\|\mathbf{x}\|_2\end{aligned}}
            \end{itemize}
      \vItem
            Induce \textbf{metric} \iMbox{d(x,y) = \lVert y - x \rVert}, has
            additional properties:

            \begin{itemize}

                  \vItem
                        Translation invariance: \iMbox{d(x+w,y+w) = d(x,y)}
                  \vItem
                        Scaling: \iMbox{d(\lambda x,\lambda y) = |\lambda|d(x,y)}
            \end{itemize}
\end{itemize}

\subsection*{Matrix norms}

\begin{itemize}

      \vItem
            \textbf{Matrix norms} are such that:
            \iMbox{\lVert A \rVert = 0 \iff A = \mathbf{0}},
            \iMbox{|\lambda A|=|\lambda| \lVert A \rVert},
            \iMbox{\|A+B\| \leq\|A\|+\|B\|}

            \begin{itemize}

                  \vItem
                        Matrices \iMbox{\mathbb{F}^{m \times n}} are a vector space so
                        \textbf{matrix norms \emph{are} vector norms}, all results apply
            \end{itemize}
      \vItem
            \textbf{Sub-multiplicative matrix norm} (assumed by default) is also
            such that
            \iMbox{\lVert AB \rVert \leq \lVert A \rVert \lVert B \rVert}
      \vItem
            Common matrix norms, for some
            \iMbox{\mathbf{A} \in \mathbb{R}^{m \times n}}:

            \begin{itemize}

                  \vItem
                        \iMbox{\|\mathbf{A}\|_1=\max _j\left\|\mathbf{A}_{\ast j}\right\|_1}
                  \vItem
                        \iMbox{\|\mathbf{A}\|_2=\sigma_1(\mathbf{A})}
                        i.e. largest singular value of \iMbox{A} \emph{(square-root of largest eigenvalue of \iMbox{A^TA} or
                              \iMbox{AA^T})}
                  \vItem
                        \iMbox{\|\mathbf{A}\|_{\infty} =\max_i \left\|\mathbf{A}_{i\ast}\right\|_1},
                        note that
                        \iMbox{\|\mathbf{A}\|_1=\left\|\mathbf{A}^T\right\|_{\infty}}
                  \vItem
                        \textbf{Frobenius norm}:
                        \iMbox{\ds \|\mathbf{A}\|_F=\sqrt{\sum_{i=1}^m \sum_{j=1}^n\left|\mathbf{A}_{i j}\right|^2}}
            \end{itemize}
      \vItem
            A matrix norm \iMbox{\|\cdot\|} on \iMbox{\mathbb{R}^{m \times n}} is
            \textbf{consistent} with the vector norms \iMbox{\|\cdot\|_a} on
            \iMbox{\mathbb{R}^n} and \iMbox{\|\cdot\|_b} on \iMbox{\mathbb{R}^m}
            if

            \begin{itemize}

                  \vItem
                        for all \iMbox{\mathbf{A} \in \mathbb{R}^{m \times n}} and
                        \iMbox{\mathbf{x} \in \mathbb{R}^n} =>
                        \iMbox{\|\mathbf{A} \mathbf{x}\|_b \leq\|\mathbf{A}\|\|\mathbf{x}\|_a}
                  \vItem
                        If \iMbox{a=b,\|\cdot\|} is \textbf{compatible} with
                        \iMbox{\|\cdot\|_a}
                  \vItem
                        Frobenius norm is \textbf{consistent} with \iMbox{\ell_{2}} norm
                        => \iMbox{\ds \|A v\|_2 \leq\|A\|_F\|v\|_2}
            \end{itemize}
      \vItem
            For a vector norm \iMbox{\|\cdot\|} on \iMbox{\mathbb{R}^n}, the
            \textbf{subordinate matrix norm} \iMbox{\|\cdot\|} on
            \iMbox{\mathbb{R}^{m \times n}} is
            \iMbox{\begin{aligned}\|\mathbf{A}\| & =\max \left\{\|\mathbf{A} \mathbf{x}\|: \mathbf{x} \in \mathbb{R}^n,\|\mathbf{x}\|=1\right\}                          \\
                              & =\max \left\{\frac{\|\mathbf{A} \mathbf{x}\|}{\|\mathbf{x}\|}: \mathbf{x} \in \mathbb{R}^n, \mathbf{x} \neq 0\right\} \\
                              & =\max \left\{\|\mathbf{A} \mathbf{x}\|: \mathbf{x} \in \mathbb{R}^n,\|\mathbf{x}\| \leq 1\right\}\end{aligned}}

      \vItem
            Vector norms are \textbf{compatible} with their \textbf{subordinate
                  matrix norms}
      \vItem
            For \iMbox{p=1,2,\infty} \textbf{matrix norm} \iMbox{\|\cdot\|_p} is
            \textbf{subordinate} to the vector norm \iMbox{\|\cdot\|_p} \emph{(and
                  thus \textbf{compatible} with)}
\end{itemize}

\subsection*{Properties of matrices}

Consider \iMbox{A \in \mathbb{R}^{m \times n}}

If \iMbox{Ax = x} for all \iMbox{x} then \iMbox{A = I}

For square \iMbox{A}, the \textbf{trace of \iMbox{A}} is the
\textbf{sum if its diagonals}, i.e. \iMbox{\mathrm{tr}(A)}

\hSep % ---

\iMbox{A} is symmetric \textbf{iff} \iMbox{A = A^{T}}; \iMbox{A} is Hermitian, iff \iMbox{A = A^{\dagger}}, i.e. its equal
to its conjugate-transpose

\begin{itemize}

      \vItem
            \iMbox{AA^{T}} and \iMbox{A^{T}A} are \textbf{symmetric} \emph{(and
                  \textbf{positive semi-definite})}
      \vItem
            For real matrices, \textbf{Hermitian/symmetric} are equivalent
            conditions
      \vItem
            Every eigenvalue \iMbox{\lambda_{i}} of \textbf{Hermitian} matrices
            is real

            \begin{itemize}

                  \vItem
                        geometric multiplicity of \iMbox{\lambda_{i}} = geometric multiplicity of \iMbox{\lambda_{i}}
                  \vItem
                        eigenvectors \iMbox{\mathbf{x}_{1},\mathbf{x}_{2}} associated
                        to distinct eigenvalues \iMbox{\lambda_{1},\lambda_{2}} are
                        \textbf{orthogonal},
                        i.e. \iMbox{\mathbf{x}_{1} \perp \mathbf{x}_{2}}
            \end{itemize}
\end{itemize}

\hSep % ---

\iMbox{A} is triangular \textbf{iff} all entries~above
\emph{(\textbf{lower-triangular})} or below
\emph{(\textbf{upper-triangular})} the main diagonal are \textbf{zero}

\begin{itemize}
      \vItem
            \textbf{Determinant} =>
            \iMbox{\left|A\right| = \prod_{i}a_{ii}}, i.e. the product of
            diagonal elements
\end{itemize}

\hSep % ---

\iMbox{A} is diagonal \textbf{iff} \iMbox{A_{ij} = 0, i \neq j},
i.e. if all off-diagonal entries are zero

\begin{itemize}
      \vItem
            Written as
            \iMbox{\mathrm{diag}_{m \times n}(\mathbf{a}) = \mathrm{diag}_{m \times n}(a_{1},\dots,a_{p}), p = \min(m,n)},
            where
            \iMbox{\mathbf{a} = [a_{1},\dots,a_{p}]^{T} \in \mathbb{R}^{p}}
            diagonal entries of \iMbox{A}
      \vItem
            For \iMbox{x \in \mathbb{R}^{n}},
            \iMbox{\begin{aligned}
                        Ax & = \mathrm{diag}_{m \times n}(a_{1},\dots,a_{p})[x_{1} \dots x_{n}]^{T} \\
                           & = [a_{1}x_{1} \dots a_{p}x_{p} \ \ 0 \dots 0]^{T} \in \mathbb{R}^{m}
                  \end{aligned}}
            \emph{(if \iMbox{p=m} those tail-zeros don't exist)}
      \vItem
            \iMbox{\mathrm{diag}_{m \times n}(\mathbf{a}) + \mathrm{diag}_{m \times n}(\mathbf{b}) = \mathrm{diag}_{m \times n}(\mathbf{a}+\mathbf{b})}
      \vItem
            Consider
            \iMbox{\mathrm{diag}_{n \times k}(c_{1},\dots,c_{q}), q = \min(n,k)},
            then
            \iMbox{\begin{aligned}
                         & \mathrm{diag}_{m \times n}(a_{1},\dots,a_{p})\mathrm{diag}_{n \times k}(c_{1},\dots,c_{q})       \\
                         & = \mathrm{diag}_{m \times k}(a_{1}c_{1},\dots,a_{r}c_{r}, 0,\dots,0) = \mathrm{diag}(\mathbf{s})
                  \end{aligned}}

            \begin{itemize}

                  \vItem
                        Where \iMbox{r = \min(p,q) = \min(m,n,k)}, and
                        \iMbox{\mathbf{s} \in \mathbb{R}^{s}, s = \min(m,k)}
            \end{itemize}
      \vItem
            \textbf{Inverse of square-diagonals} =>
            \iMbox{\mathrm{diag}(a_{1},\dots,a_{n})^{-1} = \mathrm{diag}(a_{1}^{-1},\dots,a_{n}^{-1})},
            i.e. diagonals \textbf{cannot be zero} \emph{(division by zero undefined)}
            \tcbbreak
      \vItem
            \textbf{Determinant of square-diagonals} =>
            \iMbox{\left|\mathrm{diag}(a_{1},\dots,a_{n})\right| = \prod_{i}a_{i}}
            \emph{(since they are technically triangular matrices)}
\end{itemize}

\hSep % ---

The \textbf{(column) rank} of \iMbox{A} is number of linearly
independent columns, i.e.~\iMbox{\mathrm{rk}(A)}

\begin{itemize}

      \vItem
            I.e. its the \textbf{number of pivots} in \textbf{row-echelon-form}

            \begin{itemize}

                  \vItem
                        I.e. its the \textbf{dimension} of the \textbf{column-space}
                        \iMbox{\mathrm{rk}(A) = \dim(\mathrm{C}(A))}
                  \vItem
                        I.e. its the \textbf{dimension} of the \textbf{image-space}
                        \iMbox{\mathrm{rk}(A) = \dim(\mathrm{im}(f_{A}))} of linear map
                        \iMbox{f_{A}(x) = Ax}
            \end{itemize}
      \vItem
            The \textbf{(row) rank} of \iMbox{A} is number of linearly
            independent rows
      \vItem
            The \textbf{row/column ranks} are \textbf{always the same}, hence
            \iMbox{\mathrm{rk}(A) = \dim(\mathrm{C}(A)) = \dim(\mathrm{R}(A)) =  \dim(\mathrm{C}(A^{T})) = \mathrm{rk}(A^{T})}
      \vItem
            \iMbox{A} is full-rank \textbf{iff}
            \iMbox{\mathrm{rk}(A) = \min(m,n)}, i.e.~its as linearly independent
            as possible
\end{itemize}

\hSep % ---

Two matrices
\iMbox{\mathbf{A}, \widetilde{\mathbf{A}} \in \mathbb{R}^{m \times n}}
are \textbf{equivalent} if there exist two invertible matrices
\iMbox{\mathbf{P} \in} \iMbox{\mathbb{R}^{m \times m}} and
\iMbox{\mathbf{Q} \in \mathbb{R}^{n \times n}} such that
\iMbox{\mathbf{A}=\mathbf{P} \widetilde{\mathbf{A}} \mathbf{Q}^{-1}}

Two matrices
\iMbox{\mathbf{A}, \widetilde{\mathbf{A}} \in \mathbb{R}^{n \times n}}
are \textbf{similar} if there exists an invertible matrix
\iMbox{\mathbf{P} \in} \iMbox{\mathbb{R}^{n \times n}} such that
\iMbox{\mathbf{A}=\mathbf{P} \widetilde{\mathbf{A}} \mathbf{P}^{-1}}

\begin{itemize}

      \vItem
            \textbf{Similar} matrices are equivalent, with
            \iMbox{\mathbf{Q} = \mathbf{P}}
\end{itemize}

\iMbox{A} is diagonalisable \textbf{iff} \iMbox{A} is similar to some
diagonal matrix \iMbox{D}


\subsection*{Properties of determinants}

\begin{itemize}

      \vItem
            Consider \iMbox{A \in \mathbb{R}^{n \times n}}, then
            \iMbox{{A_{ij}}' \in \mathbb{R}^{(n-1) \times (n-1)}} the
            \textbf{\iMbox{(i,j)}-minor matrix} of \iMbox{A}, obtained by deleting
            \textbf{\iMbox{i}-th row} and \textbf{\iMbox{j}-th column} from
            \iMbox{A}
      \vItem
            Then we define \textbf{determinant} of \iMbox{A},
            i.e. \iMbox{\det(A)=\lvert A \rvert}, as

            \begin{itemize}

                  \vItem
                        \iMbox{\ds \det(A) = \sum_{k=1}^{n} (-1)^{i+k} A_{ik} \det({A_{ik}}')},
                        i.e. expansion along \textbf{\iMbox{i}-th row} *(for any \iMbox{i})
                  \vItem
                        \iMbox{\ds \det(A) = \sum_{k=1}^{n} (-1)^{k+j} A_{kj} \det({A_{kj}}')},
                        i.e. expansion along \textbf{\iMbox{j}-th column} \emph{(for any
                              \iMbox{j})}
            \end{itemize}
      \vItem
            When \iMbox{\det(A) = 0} we call \iMbox{A} a \textbf{singular matrix}
      \vItem
            Common determinants

            \begin{itemize}

                  \vItem
                        For \iMbox{n=1}, \iMbox{\det(A) = A_{11}}
                  \vItem
                        For \iMbox{n=2}, \iMbox{\det(A) = A_{11}A_{22}- A_{12}A_{21}}
                  \vItem
                        \iMbox{\det(\mathbf{I}_{n}) = 1}
            \end{itemize}
      \vItem
            \textbf{Multi-linearity in columns/rows}: if
            \iMbox{\ds A = [a_{1}|\dots| a_{j} | \dots|a_{n}] = [a_{1}|\dots| \lambda x_{j} + \mu y_{j}| \dots|a_{n}]}
            then
            \iMbox{\ds \begin{aligned}\det(A) = & \lambda \det\left( [a_{1}|\dots| x_{j} | \dots|a_{n}] \right)
               \\ &+ \mu \det\left( [a_{1}|\dots| y_{j} | \dots|a_{n}] \right)\end{aligned}}

            \begin{itemize}

                  \vItem
                        And the exact same linearity property for \textbf{rows}
                  \vItem
                        \emph{Immediately} leads to: \iMbox{|A| = |A^{T}|},
                        \iMbox{|\lambda A| = \lambda^{n}|A|}, and \iMbox{|AB|=|BA| = |A||B|}
                        \emph{(for any \iMbox{B \in \mathbb{R}^{n \times n}})}
            \end{itemize}
      \vItem
            \textbf{Alternating}: if any \textbf{two columns} of \iMbox{A} are
            \textbf{equal} \emph{(or any \textbf{two rows} of \iMbox{A} are
                  \textbf{equal})}, then \iMbox{|A| = 0} (its singular)

            \begin{itemize}

                  \vItem
                        \emph{Immediately} from this (and \textbf{multi-linearity})
                        => if \textbf{columns} \emph{(or \textbf{rows})} are
                        linearly-dependent (\emph{some} are linear combinations of
                        \emph{others}) then \iMbox{|A| = 0}
                  \vItem
                        Stated in other terms =>
                        \iMbox{\mathrm{rk}(A) < n \iff |A| = 0} \textless=>
                        \iMbox{\mathrm{RREF}(A) \neq I_{n} \iff |A| = 0} \emph{(reduced
                              row-echelon-form)} \textless=>
                        \iMbox{\mathrm{C}(A) \neq \mathbb{R}^n \iff |A| = 0}
                        \emph{(column-space)}
                  \vItem
                        For more equivalence to the above, see invertible matrix theorem
            \end{itemize}
      \vItem
            Interaction with \textbf{EROs}/\textbf{ECOs}:

            \begin{itemize}

                  \vItem
                        \textbf{Swapping} rows/columns \textbf{flips the sign}
                  \vItem
                        \textbf{Scaling} a row/column by \iMbox{\lambda \neq 0} will
                        \textbf{scale the determinant} by \iMbox{\lambda} \emph{(by
                              multi-linearity)}

                        \begin{itemize}

                              \vItem
                                    Remember to \textbf{scale by \iMbox{\lambda^{-1}}} to
                                    \textbf{maintain equality},
                                    i.e. \iMbox{\ds \det(A) = \lambda^{-1} \det\left( [a_{1}|\dots| \lambda a_{i} | \dots|a_{n}] \right)}
                        \end{itemize}
                  \vItem
                        Invariant under \textbf{addition} of rows/columns
            \end{itemize}
      \vItem
            \textbf{Link to invertable matrices} =>
            \iMbox{|A^{-1}| = |A|^{-1}} which means
            \iMbox{A \ \text{is invertible} \iff |A| \neq 0}, i.e. \textbf{singular matrices} are
            \textbf{not invertible}
      \vItem
            For block-matrices:

            \begin{itemize}

                  \vItem
                        \iMbox{\det {\begin{pmatrix}A & B \\ \mathbf{0} & D\end{pmatrix} = \det(A)\det(B) = \det \begin{pmatrix}A & \mathbf{0} \\ C & D\end{pmatrix}}}
                  \vItem
                        \iMbox{\begin{aligned}
                                    \det
                                    {\begin{pmatrix}A & B \\ C & D\end{pmatrix}} & = \det(A)\det(D - CA^{-1}B) \\
                                                                                 & = \det(D)\det(A - BD^{-1}C)
                              \end{aligned}}
                        if \iMbox{A} or \iMbox{D} are invertible, \emph{respectively}
            \end{itemize}
      \vItem
            \textbf{Sylvester's determinant theorem:}
            \iMbox{\operatorname{det}\left(I_m+A B\right)=\operatorname{det}\left(I_n+B A\right)}
      \vItem
            \textbf{Matrix determinant lemma}:

            \begin{itemize}

                  \vItem
                        \iMbox{\operatorname{det}\left(\mathbf{A}+\mathbf{u} \mathbf{v}^{T}\right)=\left(1+\mathbf{v}^{T} \mathbf{A}^{-1} \mathbf{u}\right) \operatorname{det}(\mathbf{A})}
                  \vItem
                        \iMbox{\operatorname{det}\left(\mathbf{A}+\mathbf{U} \mathbf{V}^{T}\right)=\operatorname{det}\left(\mathbf{I}_{m}+\mathbf{V}^{T} \mathbf{A}^{-1} \mathbf{U}\right) \operatorname{det}(\mathbf{A})}
                  \vItem
                        \iMbox{\operatorname{det}\left(\mathbf{A}+\mathbf{U} \mathbf{W} \mathbf{V}^{T}\right)=\operatorname{det}\left(\mathbf{W}^{-1}+\mathbf{V}^{T} \mathbf{A}^{-1} \mathbf{U}\right) \operatorname{det}(\mathbf{W}) \operatorname{det}(\mathbf{A})}
            \end{itemize}
\end{itemize}

\subsection*{Tricks for computing
      determinant}

\begin{itemize}

      \vItem
            If block-triangular matrix then apply
            \iMbox{\det \begin{pmatrix}A & B \\ 0 & D\end{pmatrix} = \det(A)\det(B)}
      \vItem
            If \emph{close} to triangular matrix apply \textbf{EROs}/\textbf{ECOs}
            to get it there, then its just product of diagonals
      \vItem
            If \textbf{Cholesky/LU/QR} is possible and cheap then do it, then
            apply \iMbox{\lvert AB \rvert = |A||B|}
            \tcbbreak
      \vItem
            If all else fails, try to find row/column with \textbf{MOST} zeros

            \begin{itemize}

                  \vItem
                        Perform minimal \textbf{EROs}/\textbf{ECOs} to get that row/column
                        to be \textbf{all-but-one} zeros

                        \begin{itemize}

                              \vItem
                                    Don't forget to keep track of sign-flipping \& scaling-factors
                        \end{itemize}
                  \vItem
                        Do Laplace expansion along \emph{that} row/column =>
                        notice \textbf{all-but-one} minor matrix determinants \textbf{go to
                              zero}
            \end{itemize}
\end{itemize}

\subsection*{Representing EROs/ECOs as transformation matrices}


For \iMbox{A \in \mathbb{R}^{m \times n}}, suppose a sequence of:

\begin{itemize}

      \vItem
            \textbf{EROs} transform
            \iMbox{\ds A \rightsquigarrow_{\mathrm{EROs}} A'} =>
            there is matrix \iMbox{R} s.t. \iMbox{RA = A'}
      \vItem
            \textbf{ECOs} transform
            \iMbox{\ds A \rightsquigarrow_{\mathrm{ECOs}} A'} =>
            there is matrix \iMbox{C} s.t. \iMbox{AC = A'}
      \vItem
            Both transform
            \iMbox{\ds A \rightsquigarrow_{\mathrm{EROs+ECOs}} A'}
            => there are matrices \iMbox{R,C} s.t. \iMbox{RAC = A'}
\end{itemize}

\hSep % ---

\textbf{FORWARD:} to compute these \textbf{transformation matrices}:

\begin{itemize}

      \vItem
            Start with \iMbox{[I_{m} \ | \ A \ | \ I_{n}]}, i.e.~\iMbox{A} and
            identity matrices
      \vItem
            For every \textbf{ERO} on \iMbox{A}, do the same to \textbf{LHS}
            (i.e.~\iMbox{I_{m}})
      \vItem
            For every \textbf{ECO} on \iMbox{A}, do the same to \textbf{RHS}
            (i.e.~\iMbox{I_{n}})
      \vItem
            Once done, you should get
            \iMbox{\ds [I_{m} \ | \ A \ | \ I_{n}] \rightsquigarrow [R \ | \ A' \ | \ C]}
            with \iMbox{RAC = A'}
\end{itemize}

\hSep % ---

If the sequences of \textbf{EROs} and \textbf{ECOs} were
\iMbox{\ds R_{1},\dots,R_{\lambda}} and
\iMbox{\ds C_{1},\dots,C_{\mu}} \emph{respectively}

\begin{itemize}

      \vItem
            \iMbox{\ds R = R_{\lambda} \cdots R_{1}} and
            \iMbox{\ds C = C_{1} \cdots C_{\mu}}, so
            \iMbox{\ds (R_{\lambda} \cdots R_{1})A(C_{1} \cdots C_{\mu}) = A'}
      \vItem
            \iMbox{\ds R^{-1} = R_{1}^{-1} \cdots R_{\lambda}^{-1}} and
            \iMbox{\ds C^{-1} = C_{\mu}^{-1} \cdots C_{1}^{-1}}, where
            \iMbox{\ds R_{i}^{-1},  C_{j}^{-1}} are \textbf{inverse EROs/ECOs}
            \emph{respectively}
\end{itemize}

\hSep % ---

\textbf{BACKWARD:} once \iMbox{\ds R_{1},\dots,R_{\lambda}} and
\iMbox{\ds C_{1},\dots,C_{\mu}} for which \iMbox{RAC = A'} are
\textbf{known}, starting with \iMbox{[I_{m} \ | \ A \ | \ I_{n}]}

\begin{itemize}

      \vItem
            For \iMbox{i = 1 \to \lambda} perform \iMbox{\ds R_{i}} on
            \iMbox{A}, perform \iMbox{\ds R_{\lambda -i+1}^{-1}} on \textbf{LHS}
            (i.e.~\iMbox{I_{m}})
      \vItem
            For \iMbox{j = 1 \to \mu} perform \iMbox{\ds C_{j}} on \iMbox{A},
            perform \iMbox{\ds C_{\mu -j + 1}^{-1}} on \textbf{RHS}
            (i.e.~\iMbox{I_{n}})
      \vItem
            You should get
            \iMbox{\ds [I_{m} \ | \ A \ | \ I_{n}] \rightsquigarrow [R^{-1} \ | \ A' \ | \ C^{-1}]}
            with \iMbox{\ds A = R^{-1} {A'} C^{-1}}
\end{itemize}

\hSep % ---

You can mix-and-match the \textbf{forward/backward} modes

\begin{itemize}

      \vItem
            i.e.~inverse operations in inverse order for one, and operations in
            normal order for the other
      \vItem
            e.g.~you can do
            \iMbox{\ds [I_{m} \ | \ A \ | \ I_{n}] \rightsquigarrow [R^{-1} \ | \ A' \ | \ C]}
            to get \iMbox{AC = R^{-1}A'} => useful for LU
            factorization
\end{itemize}


\subsection*{Eigen-values/vectors}

\begin{itemize}

      \vItem
            Consider \iMbox{A \in \mathbb{R}^{n \times n}}, non-zero
            \iMbox{\mathbf{x} \in \mathbb{C}^{n}} is an \textbf{eigenvector} with
            \textbf{eigenvalue} \iMbox{\lambda \in \mathbb{C}} for \iMbox{A} if
            \iMbox{A \mathbf{x} = \lambda \mathbf{x}}

            \begin{itemize}

                  \vItem
                        If \iMbox{A \mathbf{x} = \lambda \mathbf{x}} then
                        \iMbox{A (k\mathbf{x}) = \lambda (k\mathbf{x})} for
                        \iMbox{k \neq 0}, i.e.~\iMbox{k \mathbf{x}} is also an
                        \textbf{eigenvector}
                  \vItem
                        \iMbox{A} has at most \iMbox{n} distinct \textbf{eigenvalues}
            \end{itemize}
      \vItem
            The set of all eigenvectors associated with eigenvalue \iMbox{\lambda}
            is called \textbf{eigenspace} \iMbox{E_{\lambda}} of \iMbox{A}

            \begin{itemize}

                  \vItem
                        \iMbox{E_{\lambda} = \ker(A - \lambda I)}
                  \vItem
                        The \textbf{geometric multiplicity} of \iMbox{\lambda} is
                        \iMbox{\dim(E_{\lambda}) = \dim(\ker(A - \lambda I))}
            \end{itemize}
      \vItem
            The \textbf{spectrum}
            \iMbox{Sp(A) = \{ \lambda_{1},\dots,\lambda_{n} \}} of \iMbox{A} is
            the set of all eigenvalues of \iMbox{A}
      \vItem
            The \textbf{characteristic polynomial} of \iMbox{A} is
            \iMbox{P(\lambda) = \lvert A - \lambda I \rvert = \sum_{i=0}^{n} a_{i}\lambda^{i}}

            \begin{itemize}

                  \vItem
                        \iMbox{a_{0} = |A|}, \iMbox{a_{n-1} = (-1)^{n-1} \mathrm{tr}(A)},
                        \iMbox{a_{n} = (-1)^{n}}
                  \vItem
                        \iMbox{\lambda \in \mathbb{C}} is eigenvalue of \iMbox{A}
                        \textbf{iff} \iMbox{\lambda} is a root of \iMbox{P(\lambda)}
                  \vItem
                        The \textbf{algebraic multiplicity} of \iMbox{\lambda} is the
                        \textbf{number of times} it is \textbf{repeated as root} of
                        \iMbox{P(\lambda)}
                  \vItem
                        \iMbox{1} \iMbox{\leq \text{geometric multiplicity of} \ \lambda} \iMbox{\leq \text{algebraic multiplicity of} \ \lambda}
            \end{itemize}
      \vItem
            Let \iMbox{\lambda_{1},\dots,\lambda_{n} \in \mathbb{C}} be
            \emph{(potentially non-distinct)} \textbf{eigenvalues} of \iMbox{A},
            with \iMbox{\mathbf{x}_{1},\dots, \mathbf{x}_{n} \in \mathbb{C}^{n}}
            their \textbf{eigenvectors}

            \begin{itemize}

                  \vItem
                        \iMbox{\mathrm{tr}(A) = \sum_{i} \lambda_{i}} and
                        \iMbox{\mathrm{\det}(A) = \prod_{i} \lambda_{ij}}
                  \vItem
                        \iMbox{A} is diagonalisable \textbf{iff} there exist a basis of
                        \iMbox{\mathbb{R}^{n}} consisting of
                        \iMbox{\mathbf{x}_{1},\dots, \mathbf{x}_{n}}
                  \vItem
                        \iMbox{A} is diagonalisable \textbf{iff} \iMbox{r_{i} = g_{i}},
                        where \iMbox{r_{i} = \text{geometric multiplicity of} \ \lambda_{i}}
                        and \iMbox{g_{i} = \text{geometric multiplicity of} \ \lambda_{i}}
                  \vItem
                        \textbf{Eigenvalues} of \iMbox{A^{k}} are
                        \iMbox{\lambda_{1},\dots,\lambda_{n}}
            \end{itemize}
      \vItem
            Let \iMbox{P = [\mathbf{x}_{1}|\dots|\mathbf{x}_{n}]}, then
            \iMbox{AP = [\lambda_{1}\mathbf{x}_{1}|\dots|\lambda_{n}\mathbf{x}_{n}] = [\mathbf{x}_{1}|\dots|\mathbf{x}_{n}] \mathrm{diag}(\lambda_{1},\dots,\lambda_{n}) = PD}
            => if \iMbox{P^{-1}} exists then

            \begin{itemize}

                  \vItem
                        \iMbox{A = PDP^{-1}}, i.e.~\iMbox{A} is diagonalisable
                  \vItem
                        \iMbox{P = \mathbf{I}_{EB}} is \textbf{change-in-basis} matrix for
                        basis
                        \iMbox{B = \langle \mathbf{x}_{1},\dots,\mathbf{x}_{n} \rangle} of
                        eigenvectors
                  \vItem
                        If \iMbox{A = \mathbf{F}_{EE}} is transformation-matrix of linear
                        map \iMbox{f}, then
                        \iMbox{\mathbf{F}_{EE} = \mathbf{I}_{EB} \mathbf{F}_{BB} \mathbf{I}_{BE}}
            \end{itemize}
      \vItem
            \textbf{Spectral theorem}: if \iMbox{A} is Hermitian then
            \iMbox{P^{-1}} exists:

            \begin{itemize}

                  \vItem
                        If \iMbox{\mathbf{x}_{i},\mathbf{x}_{j}} associated to different
                        eigenvalues then \iMbox{\mathbf{x}_{i} \perp \mathbf{x}_{j}}
                        \tcbbreak
                  \vItem
                        If associated to same eigenvalue \iMbox{\lambda} then
                        \textbf{eigenspace} \iMbox{E_{\lambda}} has spanning-set
                        \iMbox{\ds \{ \mathbf{x}_{\lambda_{i}},\dots \}}

                        \begin{itemize}

                              \vItem
                                    \iMbox{\mathbf{x}_{1},\dots,\mathbf{x}_{n}} are linearly
                                    independent => apply Gram-Schmidt
                                    \iMbox{\ds \mathbf{q}_{\lambda_{i}},\dots \leftarrow \mathbf{x}_{\lambda_{i}},\dots}
                              \vItem
                                    Then \iMbox{\ds \{ \mathbf{q}_{\lambda_{i}},\dots \}} is
                                    orthonormal basis (ONB) of \iMbox{E_{\lambda}}
                        \end{itemize}
                  \vItem
                        \iMbox{Q = \langle \mathbf{q}_{1},\dots,\mathbf{q}_{n} \rangle} is
                        an ONB of \iMbox{\mathbb{R}^{n}} =>
                        \iMbox{\mathbf{Q} = [\mathbf{q}_{1}|\dots|\mathbf{q}_{n}]} is
                        orthogonal matrix i.e.~\iMbox{\mathbf{Q}^{-1} = \mathbf{Q}^{T}}
                  \vItem
                        \iMbox{\mathbf{q}_{1},\dots,\mathbf{q}_{n}} are still eigenvectors
                        of \iMbox{A} => \iMbox{A = \mathbf{Q}D\mathbf{Q}^{T}}
                        \emph{(\textbf{spectral decomposition})}
                  \vItem
                        \iMbox{A = \mathbf{Q}D\mathbf{Q}^{T}} can be interpreted as scaling
                        in direction of its eigenvectors:

                        \begin{enumerate}
                              \vItem
                                    Perform a succession of reflections/planar rotations to change
                                    coordinate-system
                              \vItem
                                    Apply scaling by \iMbox{\lambda_{i}} to each dimension
                                    \iMbox{\mathbf{q}_{i}}
                              \vItem
                                    Undo those reflections/planar rotations
                        \end{enumerate}
            \end{itemize}
\end{itemize}